\documentclass[aspectratio=1610]{beamer}

\usepackage[T1]{fontenc}
\usepackage[utf8]{inputenc}
\usepackage[ngerman]{babel}
\usepackage{lmodern}
\usepackage{tikz}
\usetikzlibrary{intersections, positioning}

% customize 

%\useoutertheme{infolines}
%\useoutertheme[height=0pt, right, hideothersubsections]{sidebar}
%\useinnertheme{rectangles}
%\setbeamertemplate{headline}{}
\usetheme{MyTheme}

\title{Software Architektur}
\subtitle{einführung und Best Practices}
\author{Moamen Hraden}
\date{\today}

\begin{document}


	\begin{frame}[plain]
    	\titlepage
	\end{frame} 
	
	
	\section{Agenda}
	\begin{frame}{Agenda}
    	\tableofcontents
	\end{frame}

	
	
	
	\section{Einführung}	
	\begin{frame}{Einführung}
		\begin{columns}
			\centering
			\begin{column}{.4\paperwidth}
				\centering
				\begin{block}{Definition}
					dieser box wurde nur zum Demonstrieren eingefügt.
				\end{block}
			\end{column}
			\begin{column}{.4\paperwidth}
				\centering
				\begin{tikzpicture}
					\node[circle, inner sep=1cm, fill=orange, draw=black] at (0, 0) {this is circle}; 
				\end{tikzpicture}
			\end{column}
		\end{columns}
	\end{frame}
	
	\section{Einführung(2)}
	\begin{frame}{Einführung(2)}
		\begin{alertblock}{Definition}
			dieser box wurde nur zum Demonstrieren eingefügt.
		\end{alertblock}
	\end{frame}
	
	\section{Columns}
	\begin{frame}{Columns}
		\begin{columns}
			\begin{column}{5cm}
				Linke Spalte
			\end{column}
			\begin{column}{5cm}
				Rechte Spalte
			\end{column}
		\end{columns}
	\end{frame}
	
	\begin{frame}{overlays}
		\begin{enumerate}
			\item microservices 
			\pause
			\item Domain Driven Design
			\pause 
			\item Monolithic
			\pause
			\item Service Oriented
		\end{enumerate}
	\end{frame}
	
	\begin{frame}{simple graph}
		\begin{tikzpicture}
			\draw[step=0.5cm] (0,0) grid (1.4, 1.4); 
			\draw (0,0) -- (1.5, 0); 
			\draw (0,0) -- (0, 1.5);
		\end{tikzpicture}
	\end{frame}
	
	\begin{frame}{Graphen}
		\begin{tikzpicture}[scale=2]
			\draw[red, thick] (0,0) -- (1,1) -- (2,0) -- (0,0); 
			\draw[dashed] (3,0) rectangle (5,1) ; 
			\draw (7, 0.5) circle (0.5); 	
		\end{tikzpicture}
	\end{frame}
	
	\begin{frame}{Knoten}
		\begin{tikzpicture}
			\node[draw] at (0,0) (k1) {knoten1}; 
		\end{tikzpicture}
	\end{frame}
	
	
	\begin{frame}{beispiel-1}
		\begin{center}
		\begin{tikzpicture}
			\fill[green] (0,0) circle (2); 
			\draw[dashed, orange] (0,0) circle (4); 
			\fill[orange] (0, 4) circle (0.3); 
			\draw[->] (0,3.7) --node[right]{$\vec{F_g}$} (0,2); 
			\draw[->] (-0.3,4) --node[above]{$\vec{F_g}$} (-1,3); 
			
		\end{tikzpicture}
		\end{center}	
	\end{frame}
	
	\begin{frame}{beispiel-2}
		\begin{center}
		\begin{tikzpicture}
				
		\end{tikzpicture}
		\end{center}	
	\end{frame}
	
	
	\begin{frame}{Beispiel}
		dieser Text wird auf der Folie angezeigt. :) 
		\begin{itemize}
			\item das ist das erste Element
			\item und das ist das zweite Element
		\end{itemize}
		\begin{example}{fakulität}
			\[ f(x) = x \dot f(x-1) \]
			\[ sum(n) = \sum_{i=0}^{n} \frac{n \dot (n+1)}{n} \]
			
		\end{example}
	\end{frame}
	
	\section{flowchart}
	\begin{frame}{flowchart}
		\begin{figure}
			\centering
			\begin{tikzpicture}[node distance= 0.5cm and 0.2cm]
				\node[fill=orange, circle] (start) {}; 
				\node[fill=orange!20, rectangle, below=of start] (click) 	{Click}; 
				\node[fill=green!20, rectangle, below right=of click] (auth) {authenticated}; 
				\node[fill=red!20, rectangle, below left=of click] (notauth) {not authenticated}; 
				\node[fill=orange, circle, below=of auth, radius=3cm] (end) {}; 
				\draw[->] (start) -- (click) ; 
				\draw[->] (click) -| (auth); 
				\draw[->] (click) -| (notauth); 
				\draw[->] (auth) -- (end) ; 
				\draw[->] (notauth) -- (end); 
			\end{tikzpicture}
		\end{figure}
	\end{frame}
	
	
	
	


\end{document}